%!TEX root = ../thesis.tex
%*******************************************************************************
%****************************** Third Chapter **********************************
%*******************************************************************************

\chapter{Data Description}%
\label{ch:data}
The content of this chapter will be an overview of the starting data used in this research, as well as some general statistics about it.

\section{Source of the data}
The data considered are tweets collected from the Twitter API containing the hashtags \#cop21 and \#cop26. 
For each cop,  there are two jsonlines files, one for the tweets and one for the users involved. The fields are the ones stated in the documentation \href{https://docs.google.com/document/d/1C6DjDXXu-fHtpighvW524e6uD2eFxLOq/edit?pli=1\#heading=h.1pxezwc}{\textsuperscript{1}}; among the numerous fields, the most relevant for this work  are the following:

\begin{table}[H]
\centering
\begin{tabular}{|l|l|}
\hline
\textbf{Field} & \textbf{Description} \\ \hline
author & The ID of the author \\ \hline
author\_name & The username of the author \\ \hline
text & The text of the tweet \\ \hline
date & The creation date of the tweet \\ \hline
lang & The language of the tweet \\ \hline
conversation\_id & The ID of the conversation the tweet belongs to \\ \hline
referenced\_type & The type of the referenced tweet \\ \hline
referenced\_id & The ID of the referenced tweet \\ \hline
mentions\_name & The usernames of the mentioned users in the tweet \\ \hline
mentions\_id & The IDs of the mentioned users in the tweet \\ \hline

\end{tabular}
\caption{Description of the fields used of the tweets data}
\label{tab:my_label}
\end{table}
 The user’s file is used to map the ID of the users to their username, but when this information is not available, the user ID is treated as the username.

\section{ Data Statistics}
The data considered originates from 2 COPs: COP21, COP26. All the tweets are in English and without links or image/video content. For this work,  an ‘original tweet’ is a tweet written by a user, meaning that it is  not a retweet. Fig \ref{fig:tweets_by_date} shows the distribution of the tweets over time for both cops; most of the tweets have been tweeted while the conferences were taking place.

\begin{figure}[H]
    \centering
    \includegraphics[width=0.75\linewidth ]{Chapter3/figures/tweets_by_date_cop21.png}

    \includegraphics[width=0.75\linewidth ]{Chapter3/figures/tweets_by_date_cop26.png}
    \caption{Numeber of tweets by date for cop 21 and cop26}
    \label{fig:tweets_by_date}
\end{figure}

\paragraph{COP21}
The tweets span from January 2015 to June 2016, but 77\% of the tweets are from November and December 2021; cop26 was held between November 30th and December 12th. In the dataset, 975040 tweets were tweeted by 234389 users, of which only x tweeted an original tweet with at least one retweet; every user tweeted on average 4.16 tweets; the maximum amount of tweets a user tweeted is 9635, while 89\% of users tweeted less than five tweets.

\paragraph{COP26}
The tweets span from January 2021 to July 2022, but 81\%  of the tweets are from October and November 2021; COP26 was held between  October 31st and  November 12th. In the dataset, 1558968 tweets were tweeted by 456000 users, of which only 30195 tweeted an original tweet with at least one retweet. Every user tweeted on average 3.42 tweets, the maximum number of tweets a user tweeted is 14267, while 90\% of users tweeted less than two tweets.
\\


Fig \ref{fig:tweets_by_users} shows how most users tweeted just a few tweets (note that it is logarithmic).

\begin{figure}
    \centering
    \includegraphics[width=0.75\linewidth ]{Chapter3/figures/tweets_by_users_cop21.png}

    \includegraphics[width=0.75\linewidth ]{Chapter3/figures/tweets_by_users_cop26.png}
    \caption{Number of tweets by user for cop 21 and cop26}
    \label{fig:tweets_by_users}
\end{figure}






\begin{table}[h]
\centering
\setlength{\tabcolsep}{10pt} % Sets the horizontal padding
\renewcommand{\arraystretch}{1.5} % Sets the vertical padding
\begin{tabular}{
  |l|
  S[table-format=7.0,group-four-digits=true]|
  S[table-format=7.0,group-four-digits=true]|
  S[table-format=7.0,group-four-digits=true]|
  S[table-format=7.0,group-four-digits=true]|
}
\hline
 & {\textbf{n\_tweets}} & {\textbf{n\_retweets}} & {\textbf{n\_original}} & {\textbf{n\_original\_with\_retweets}} \\ \hline
\textbf{COP21} & 975040 & 562946 & 412094 & 138427 \\ \hline
\textbf{COP26} & 1558968 & 1191813 & 367155 & 130138 \\ \hline
\end{tabular}
\caption{Number of tweets}
\label{tab:n_tweets}
\end{table}




Fig \ref{fig:cop26_tweets_stats} depicts how the 1'558'968 are distributed: 76\% of them are retweets generated by only 130k original tweets. It is also worth noting that almost 2/3 of the original tweets have 0 retweets.

\begin{figure}
    \centering
    \includegraphics[width=0.9\linewidth]{Chapter3/figures/treemap_tweets-1.png}
    \caption{tweets of cop26}
    \label{fig:cop26_tweets_stats}
\end{figure}








% **************************** Define Graphics Path **************************
\ifpdf
    \graphicspath{{Chapter3/Figs/Raster/}{Chapter3/Figs/PDF/}{Chapter3/Figs/}}
\else
    \graphicspath{{Chapter3/Figs/Vector/}{Chapter3/Figs/}}
\fi


