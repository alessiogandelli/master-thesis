\chapter{Conclusion}
\label{ch:conclusion}
The goal of this research was to investigate the polarization of users in the climate change discussion around the Conferences of Parties, specifically focusing on the 26th Conference. Unlike similar studies, this analysis was conducted topic by topic using a multi-layer network framework. The research is aimed at answering several questions: first, identifying the most polarized topics of COP26 and comparing them with a longitudinal study involving COP21; second, examining user polarization across different topics and exploring whether users who engage in more topics are more polarized.

The first part of the analysis is concerned with the evaluation of the topic modeling techniques, using both traditional metrics such as coherence and diversity both an ad-hoc experiment. The results showed that the models exploiting neural networks architectures, in particular Bertopic, performed better than the traditional ones.

The second part was the actual analysis of the multilayer networks.
The results revealed that in the most polarized topics, both sides had nearly an equal number of users, indicating a sharp divide in opinions. This was particularly evident in the Canadian discussion on limiting the use of coal and oil, as well as in the air travel debate. Additionally, it was observed that the Canadian discussion on Fossil Fuel was not always polarized: in COP21, for instance, it was completely non-polarized, which aligns with Falkenberg's findings.

The users tended to stay on the same side of the discussion across multiple topics, but there was no correlation between the number of topics and polarization score. It is important to note that this thesis only addressed a limited set of questions using the presented methodology, focusing solely on the retweet network. However, the reply and quote networks have different properties that can be further explored, especially the reply network, which allows for the study of direct discussions between users.

The use of transformers in the topic modeling step enabled highly accurate results, which previously would have required extensive manual work (annotating tweets). The use of a multilayer network framework prevents information loss by consolidating all data into a single network. Moreover, this framework has the potential to extend beyond the study of climate change and Twitter, and can be applied to any scenario where actors engage in discussions on multiple topics.

A limitation of this work is that only English tweets have been considered and while they constitute a significant portion of the total, conducting the same analysis with other languages may yield different results. This is particularly relevant for regional languages such as Italian. Additionally, adjusting the methodology by incorporating domain knowledge about polarization geometry\cite{radicioni_networked_2021} could further enhance the findings.

Further research should investigate the growth of polarization resulting from new environmental activism that inherently exhibits polarization in its methodology, such as blocking streets or defiling monuments. These acts undoubtedly generate conversation, but it remains to be seen whether these discussions occur within echo chambers or involve individuals from opposite ends of the spectrum.



