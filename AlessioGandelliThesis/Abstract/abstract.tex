% ************************** Thesis Abstract *****************************
% Use `abstract' as an option in the document class to print only the titlepage and the abstract.
\begin{abstract}
This thesis aimed to investigate the polarization of users in the climate change discussion surrounding the Conferences of Parties, with a specific focus on the 26th Conference. Unlike previous studies, this research adopted a topic-by-topic approach using a multi-layer network framework. The objectives were twofold: first, to identify the most polarized topics of COP26 and compare them with a longitudinal study involving COP21; and second, to explore user polarization across different topics and investigate whether users engaging in more topics exhibit higher levels of polarization. 
The findings revealed that in the most polarized topics, there was an almost equal distribution of users on both sides, indicating a sharp divide in opinions. Notably, this polarization was particularly evident in the Canadian discussion on limiting the use of coal and oil, as well as the air travel debate. Additionally, it was observed that the Canadian discussion on Fossil Fuel was not consistently polarized, as it exhibited complete non-polarization during COP21, confirming the results obtained by other researches.
Furthermore, users tended to remain aligned with a specific side of the discussion across multiple topics, although there was no correlation between the number of topics and the polarization score.
\end{abstract}
